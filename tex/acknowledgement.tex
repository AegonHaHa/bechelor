\begin{Acknowledgement}{}
    \par 逝者如斯夫,不舍昼夜。人生就是不断地拿笔为一段段故事郑重开头,也不断地为另一些故事画上句号,不论它们是否圆满,人生短暂,这一辈子能叙写的故事不会太多。我很有幸,很感激,在18岁到22岁这段最美好的年华,我的故事与东大一起书写,与南京一起书写。

    \par 大学四年,俯仰之间而已。还记得大一烈日炎炎下不屈不挠的军训让我的内心一点一点坚定;还记得和文教优秀的伙伴一起办过一场场讲座、大活动,和伙伴结下深厚友谊,同时感受人文熏陶;还记得四年三次和拉丁舞社的伙伴一起登上跨年舞台,为大学留下难忘的精力;还记得一次次比赛、一次次考试让我成长\ldots 尽管四年时光匆匆从指间溜走,但是回想起来,还是又那么多点点滴滴值得我珍藏,回味一生。

    \par 写到了这一段,我也快要给自己的大学生活画上句号了。这篇论文背后的项目已经磕磕绊绊进行了五个多月,从开始不知所措到渐渐找到感觉,我经历了不少。科研的路上没有明确的路牌,在摸索中前进,也可能走入岔路,再掉头回来,踏上新的路,如此往复,坚持不懈,终\textbf{可能}有所获。这个项目到这里仅仅是一个阶段,距离最终成果还有不少距离,我仍然在路上。感谢项目组里余旻晨师兄的指导与辛勤付出,感谢项目组杨柏辰学弟为代码、测试贡献了不少力量,感谢王威老师对项目的悉心指导和很多建设性的意见,感谢余英豪师兄远程参与,感谢田黄石师兄、翁祈桢师兄、晏达师兄的宝贵建议。

    \par 回到东大,感谢我的校内指导老师肖卿俊老师的悉心指导,他带我进入科研的大门,让我参与课题组的项目,在组会上作报告锻炼我的能力,指导我们参加SDN大赛,为我写推荐信。感谢所有教过我的老师,是你们让我看到计算机科学丰富多彩的世界,激发我在这个世界继续探索的动力。感谢我的辅导员老师刘舒辰和魏敏娜老师,在我的大学生活中给予我帮助和鼓励。感谢与我并肩作战的舍友刘浩、王坤鹏、冷相宏同学,感谢2015级计科一班的同学,感谢计算机学院、东南大学。最最重要的,感谢我的父母郑群超先生和肖丽女士,你们养育我健康长大,你们的付出让我在求学之路上没有后顾之忧。

    \par 希望此篇成为我在计算机科学道路的起点,遇到困难能够不屈不挠,乘风破浪,不忘初心。与诸君共勉。

\end{Acknowledgement}