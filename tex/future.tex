\chapter{总结与展望}
\label{chp:future}

\par 本章对CW-Cache当前完成的工作进行总结,并对CW-Cache未来的研究方向与计划进行展望。

\section{工作总结}

\par 我们注意到数据密集型集群中出现的负载均衡的问题,并且在文献调研中发现,研究人员几乎都关注一般意义上的文件的负载均衡,缺少对结构化数据的文件的特别关注。本文研究发现,集群里对结构化数据(数据表)的各个列的访问热度存在较大差异,且两列的热度越高,二者被共同访问的概率也越高。基于这个发现,本文设计、分析、开发了CW-Cache系统,希望实现结构化数据在列级别的负载均衡。目前CW-Cache实现了Bundle-K方案,即将热度排在前$K$的列“捆绑”在一起进行复制,复制$r$份。本文对Bundle-K方案进行了数学建模,给定一段时间内统计的各个列的热度和SQL查询任务的数量,求出当前最优的$K$和$r$。

\par 我们在分布式内存文件系统Alluxio上实现了CW-Cache,尽量只利用Alluxio能够获得的信息,避免与上层计算框架的耦合。


\section{未来展望}

\par 在完成毕业论文的写作与答辩后,我仍将继续进行这个项目,采用Bundle-K方案的CW-Cache远远达不到完善的程度,还有很多工作于研究亟待进行,这个方案仅仅适用于一部分案例,还存在不少问题:

\begin{itemize}
    \item CW-Cache识别列是借助文件URI、偏移量和读取长度,实质上已经放弃了文件存储的结构化数据的语义信息,是和系统实现的妥协。本身Parquet文件在存储时各个列在物理上并不连续,导致在CW-Cache系统看来,每一个列是由若干“文件片段”构成的。能否在不增加上下层耦合度的情况下再次引入列的语义信息,这是否能提高系统的性能,留待探索。
    \item 考虑列的共同访问模式,当前CW-Cache系统是设置一个时间窗口,来判断相邻的被访问的列是否同属一个SQL查询任务。现在我们认为当Spark SQL解析SQL语句后,会并发地取获取需要访问的列,于是将这个窗口设置得比较小,但情况是否真的全是这样呢?此外,只有一个用户使用CW-Cache时还好,当多个用户同时使用时,结果将会不准确,我们需要想其他的办法避开。
    \item Bundle-K方案也许不够灵活,Bundle-K有两个极端特例,一个是Bundle-1,也就是把热度最高的列复制多份并分散在集群中缓存,如果这张表的访问热度倾斜很严重,那么Bundle-1可能取得好的效果;另一个是Bundle-N,也就是全表复制,相当于现有的复制方案,这个可以完全避免表内数据shuffle,但是复制成本高。我们现在的Bundle-K是这两个方案的折中,但是我们目前将这$K$列视作整体一起复制,复制的份数相同,不够灵活。Bundle-K应用在图\ref{fig:calc-dd}、\ref{fig:emul-dd}、\ref{fig:emul-wr}、\ref{fig:emul-ws}所示的表上有不错的效果,因为当$K$比较小的时候,随着$K$的上升,前$K$列组成的副本能承担大量的负载,而图~\ref{fig:calc-wr}、\ref{fig:calc-ws}的曲线接近线性增长,收益不大。
\end{itemize}

\par 此外,数据shuffling与数据放置的位置、网络通信状况、任务调度等因素均有关,非常复杂,它产生的影响难以通过数学建模进行研究。我们非常希望有一个方法能够以较小的复制开销实现负载均衡,同时能够避免shuffling,而这样的one-size-fit-all的方法几乎是不存在的,还是需要在不同方案间考虑不同场景进行取舍。